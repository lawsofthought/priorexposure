\documentclass{slides}

\title[MCMC]{Introducing Markov Chain Monte Carlo}
\author[Andrews]{Mark Andrews \& Thom Baguley}
\date{Sept 23, 2015}

\begin{document}
{
	\begin{frame}
		\titlepage
	\end{frame}
}


\begin{frame}
	\frametitle{Posterior distributions}
	\begin{itemize}

		\item In general, given observed data $D$ and a model $\Omega$, the posterior distribution over the parameters $\theta$ of the model is
		\[
			\Prob{\theta\given D, \Omega}
			= \frac{\overbrace{\Prob{D\given\theta}}^{\text{Likelihood}}\overbrace{\Prob{\theta\given\Omega}}^{\text{Prior}}}
			{\underbrace{\int\Prob{D\given\theta}\Prob{\theta\given\Omega}\ d\theta}_{\text{Marginal likelihood}}}.
		\]
		where the \emph{marginal likelihood} gives the likelihood of the model given the observed data.

	\item Given the posterior distribution $\Prob{\theta\given D, \Omega}$, our aim is often to characterise this distribution in terms of e.g. its mean, variance, etc. 
	\item Likewise, we may aim to calculate \emph{posterior predictive} distributions such as 
		\[
			\Prob{x_{\text{new}} \given D, \Omega} = \int\Prob{x_{\text{new}}\given\theta,\Omega}\Prob{\theta\given D, \Omega}\ d\theta.
		\]

	\end{itemize}
\end{frame}

\begin{frame}
	\frametitle{Sampling from posterior distributions}
	\begin{itemize}

		\item In only rare situations can we determine the characteristics of the posterior distribution, or calculate posterior predictive distributions, in closed form.
		\item However, in general, if we can draw samples from $\Prob{\theta\given D, \Omega}$ then we can approximate, e.g., the mean of the distribution by
			\[
				\langle\theta\rangle = \int\theta\Prob{\theta\given D, \Omega} \approx \frac{1}{N} \sum_{i=1}^N \tilde{\theta}_i,
			\]
		or the posterior predictive distribution by 
		\[
			\Prob{x_{\text{new}} \given D, \Omega} = \int\Prob{x_{\text{new}}\given\theta,\Omega}\Prob{\theta\given D, \Omega}\ d\theta
			\approx \frac{1}{N} \sum_{i=1}^N \Prob{x_{\text{new}}\given\tilde{\theta}_i,\Omega},
		\]
		where \[\tilde{\theta}_1, \tilde{\theta}_2 \ldots \tilde{\theta}_N\] are samples from $\Prob{\theta\given D, \Omega}$.
	\end{itemize}
\end{frame}

\begin{frame}
	\frametitle{Rejection sampling}
	\begin{itemize}
		\item Rejection sampling is one of the simplest methods for sampling from posterior distributions.
		\item Let us denote $\Prob{\theta\given D, \Omega}$ by $f(\theta)$.
		\item First we sample from $\tilde{\theta}$ from another (simpler) distribution $g(\theta)$.
		\item The distribution $g(\theta)$ can be any function so long as there exists a constant $M$ such that 
			\[ M \cdot g(\theta) \geq f(\theta),\]
			for all possible values of $\theta$.
		\item Then draw $u \sim U(0, 1)$, a random sample from a uniform distribution between $0$ and $1$.
		\item If\[ u \leq \frac{f(\tilde{\theta})}{M \cdot g(\tilde{\theta})}\]
			then keep $\tilde{\theta}$. 
		\item Continue until $N$ samples are collected.

	\end{itemize}
\end{frame}

\begin{frame}
	\frametitle{Gibbs sampling}
	\begin{itemize}
		\item In a multivariate probability distribution, e.g. 
			\[\Prob{x, y, z},\]
			the univariate conditional distributions, e.g. $\Prob{x
			\given y=\tilde{y}, z=\tilde{z}}$, may be
			straightforward to sample from.
		\item In Gibbs sampling, we set e.g. $y$ and $z$ to initial values $\tilde{y}_0$ and $\tilde{z}_0$ and then sample
			\begin{align*}
				&\tilde{x}_0 \sim \Prob{x \given y=\tilde{y}_0, z=\tilde{z}_0},\\
				&\tilde{y}_1 \sim \Prob{x \given x=\tilde{x}_0, z=\tilde{z}_0},\\
				&\tilde{z}_1 \sim \Prob{x \given x=\tilde{x}_0, y=\tilde{y}_1},
			\end{align*}
		and so on.
	\item After convergence, the samples e.g. $\{\tilde{x}_N, \tilde{y}_N, \tilde{z}_N\}$ are draws from $\Prob{x, y, z}$.
	\end{itemize}
\end{frame}

\begin{frame}
	\frametitle{Metropolis Hastings}
	\begin{itemize}	
		\item Let us denote $\Prob{\theta\given D, \Omega}$ by $f(\theta)$.
		\item We sample from a symmetric \emph{proposal} distribution $Q(\cdot\vert\cdot)$.
		\item We start with an initial $\tilde{\theta}_0$, and sample 
			\[\tilde{\theta} \sim Q(\theta\vert\tilde{\theta}_0).\]
		\item We then accept $\tilde{\theta}$ with probability 
			\[\alpha = \min\left(1.0, \frac{f(\tilde{\theta})}{f(\tilde{\theta}_0)}\right).\]
		\item After convergence, the accepted samples are draws from the distribution $f(\theta)$.
		\item For Metropolis Hastings, the distribution $f(\theta)$ need be only known up to a proportional constant.
	\end{itemize}
\end{frame}



\end{document}
