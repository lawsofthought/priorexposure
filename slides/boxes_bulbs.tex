\documentclass{mjandrews.notes.palatino}
\usepackage[misc]{ifsym}
\usepackage{tikz}
\usetikzlibrary{trees}
\usepackage{verbatim}
\usepackage{epsdice}

% Alter some LaTeX defaults for better treatment of figures:
    % See p.105 of "TeX Unbound" for suggested values.
    % See pp. 199-200 of Lamport's "LaTeX" book for details.
    %   General parameters, for ALL pages:
    \renewcommand{\topfraction}{0.9}	% max fraction of floats at top
    \renewcommand{\bottomfraction}{0.8}	% max fraction of floats at bottom
    %   Parameters for TEXT pages (not float pages):
    \setcounter{topnumber}{2}
    \setcounter{bottomnumber}{2}
    \setcounter{totalnumber}{4}     % 2 may work better
    \setcounter{dbltopnumber}{2}    % for 2-column pages
    \renewcommand{\dbltopfraction}{0.9}	% fit big float above 2-col. text
    \renewcommand{\textfraction}{0.07}	% allow minimal text w. figs
    %   Parameters for FLOAT pages (not text pages):
    \renewcommand{\floatpagefraction}{0.7}	% require fuller float pages
	% N.B.: floatpagefraction MUST be less than topfraction !!
    \renewcommand{\dblfloatpagefraction}{0.7}	% require fuller float pages

	% remember to use [htp] or [htpb] for placement


\newcommand{\dice}[1]{$\text{\Cube{#1}}$} 
\input{andrews_commands}
\usepackage{algorithmic}
\usepackage{amssymb,latexsym,amsmath,amsfonts,amscd}
\usepackage{apacite}
\usepackage{lscape,graphics,graphicx,multirow}

\DeclareSymbolFont{legacymaths}{OT1}{cmr}{m}{n}
\DeclareMathAccent{\dot}     {\mathalpha}{legacymaths}{95}
\DeclareMathAccent{\bar}     {\mathalpha}{legacymaths}{22}
\DeclareMathAccent{\tilde}     {\mathalpha}{legacymaths}{126}

\usepackage[english]{babel}
\title{The ``Boxes and Bulbs'' Problem}
\author{Mark Andrews}
\begin{document}
\maketitle
\begin{abstract}
Here, I provide a detailed account of the solution of the ``Boxes and Bulbs''
problem. The other conditional probability questions are of an identical form
to this one, and so their solutions too will be of an identical nature. If you are
having difficulty with any of these types of problems, then hopefully this
solution will provide with the key insights necessary to solve these problems.
\end{abstract}

\section{The problem}

The ``Boxes and Bulbs'' problem\marginnote{This problem is taken from \emph{Schaum's Outlines: Probability} (2nd ed., 2000), pages 87-88.} is as follows:
\begin{quotation}
{\itshape
Box A has 10 lightbulbs, of which 4 are defective.
Box B has 6 lightbulbs, of which 1 is defective.
Box C has 8 lightbulbs, of which 3 are defective.
I) If we randomly choose a box, and then randomly choose a lightbulb from that box, what is the probability that we will choose a non-defective bulb?
II) If we do choose a nondefective bulb, what is the probability it came from Box C? 
}
\end{quotation}

\section{The solution}

First, note that the solution requires the calculation of two probabilities:
\[
\Prob{\text{Bulb}=\text{Working}} \quad\text{and}\quad \Prob{\text{Box}=C\given\text{Bulb}=\text{Working}}.
\]
These are the answers to Part I and Part II, respectively. Second, if you
correctly work out the first probability, i.e.
$\Prob{\text{Bulb}=\text{Working}}$, then will have all the information
necessary to obtain the second probability, i.e.
$\Prob{\text{Box}=C\given\text{Bulb}=\text{Working}}$. The third important thing to note is that the first probability is \emph{not} this:
\[
\frac{\#(\text{Number of Working Bulbs)}}{\#(\text{Total Number of Bulbs)}}.
\]

\subsection{The red herring}

The reason why the simple calculation just mentioned does not give
$\Prob{\text{Bulb}=\text{Working}}$ is because we do not choose a bulb randomly
from the total set. We first choose one of the three boxes, with equal
probability, and then choose a bulb from within the chosen box. This makes a
difference. For example, we see that Box A has 10 bulbs altogether, while Box B
has only 6. If we chose a bulb randomly from the total number of bulbs, then we
should choose a bulb from Box A nearly twice as often as from Box B. However, this is
obviously not the case as we know we choose Box A and Box B with equal
probability.  As another example, we see there is 1 defective bulb in Box B and
there are $10+6+8=24$ bulbs together in the three boxes. However, the chance
of choosing the defective bulb in Box B is not simply $1/24$. It is the
probability of choosing Box A, i.e. $1/3$, times the probability of choosing
the one defective bulb in Box B, i.e. $1/6$, which is $1/3 \times 1/6 = 1/18$.
In general, it is necessary to think about the way the boxes and bulbs are
chosen before you can correctly go from the frequencies of bulbs and boxes to
the probabilities of bulbs and boxes. 

\subsection{The right way}

As just mentioned, to get the overall probability of, say, choosing a bulb that
is from Box A and also working, we need to calculate the probability of choosing
a working bulb from Box A and multiply this by the probability of choosing Box
A in the first place. You need to follow this same procedure for all six
possibilities, i.e. Box A and bulb is working, Box A and bulb is defective, Box
B and bulb is working, etc. This process is not too difficult, but is hopefully
made even clearer with the using the tree structure of probabilities outlined
below. 


% Set the overall layout of the tree
\tikzstyle{level 1}=[level distance=4.0cm, sibling distance=3.0cm]
\tikzstyle{level 2}=[level distance=5.5cm, sibling distance=1.5cm]

% Define styles for bags and leafs
\tikzstyle{branch} = [text width=4em, text centered]
\tikzstyle{leaf} = [circle, minimum width=3pt,fill, inner sep=0pt]

% The sloped option gives rotated edge labels. Personally
% I find sloped labels a bit difficult to read. Remove the sloped options
% to get horizontal labels. 
\begin{figure*}
\begin{tikzpicture}[grow=right, sloped]
%\tikzstyle{every node}=[font=\footnotesize]
\node[branch] {}
    child {
        node[branch] {Box C}        
           child {
                node[leaf, label=right:
                    {$\Prob{\text{C},\text{Defective}} = \frac{1}{3}\times\frac{3}{8}\approx.125$}] {}
                edge from parent
                node[below] {$\Prob{\text{Defective}\given\text{C}}=\frac{3}{8}$}
            } 
	    child {
                node[leaf, label=right:
                    {$\Prob{\text{C},\text{Working}} = \frac{1}{3}\times\frac{5}{8}\approx.208$}] {}
                edge from parent
                node[above] {$\Prob{\text{Working}\given\text{C}}=\frac{5}{8}$}
            }
            edge from parent 
            node[above] {$\Prob{C}=\frac{1}{3}$}
    }
    child {
        node[branch] {Box B}        
            child {
                node[leaf, label=right:
                    {$\Prob{\text{B},\text{Defective}} = \frac{1}{3}\times\frac{1}{6}\approx.056$}] {}
                edge from parent
                node[below] {$\Prob{\text{Defective}\given\text{B}}=\frac{1}{6}$}
            }
           child {
                node[leaf, label=right:
                    {$\Prob{\text{B},\text{Working}} = \frac{1}{3}\times\frac{5}{6}\approx.278$}] {}
                edge from parent
                node[above] {$\Prob{\text{Working}\given\text{B}}=\frac{5}{6}$}
            } edge from parent 
            node[above] {$\Prob{B}=\frac{1}{3}$}
    }
    child {
        node[branch] {Box A}        
           child {
                node[leaf, label=right:
                    {$\Prob{\text{A},\text{Defective}} = \frac{1}{3}\times\frac{4}{10}\approx.133$}] {}
                edge from parent
                node[below] {$\Prob{\text{Defective}\given\text{A}}=\frac{4}{10}$}
            } 	
           child {
                node[leaf, label=right:
                    {$\Prob{\text{A},\text{Working}} = \frac{1}{3}\times\frac{6}{10}\approx.2$}] {}
                edge from parent
                node[above] {$\Prob{\text{Working}\given\text{A}}=\frac{6}{10}$}
            }
            edge from parent 
            node[above] {$\Prob{A}=\frac{1}{3}$}
    }
    ;
\end{tikzpicture}
\end{figure*}

This tree structure gives you all the information you need to answer the questions. Note that on the right are given all the joint probabilities, which can be arranged to a joint probability table as follows:
\begin{center}
\begin{tabular}{rcc}
& Working & Defective \\
Box A & $.2$ & $.133$ \\
Box B & $.278$ & $.056$ \\
Box C & $.208$ & $.125$
\end{tabular}
\end{center}
Now it is simple to see the overall probability of choosing a working bulb. It is probability of choosing Box A and choosing a working bulb \emph{or} choosing Box B and choosing a working bulb \emph{or} choosing Box C and choosing a working bulb. This is the sum of the first column, which is $.2+.278+.208\approx.686$. 
Likewise, to get the probability of choosing Box C \emph{given} that we have chosen a working bulb, we see what proportion of the total probability of choosing a working bulb, i.e. $.686$ is from when we choose Box C and a working bulb, i.e. $.208$. In other words, it is $.208/.686\approx .304$. 

\subsection{Using probability rules}

The probabilities just mentioned can all be calculated using rules of
probability. In fact, this is precisely what is being done using the tree
structure to get the joint probability table and then summing and dividing,
etc., as appropriate. It is important to be able to see this explicitly.  

The first question asks us to calculate $\Prob{\text{Bulb}=\text{Working}}$. We know that this is equal to 
\begin{align*}
\Prob{\text{Working}} &= \Prob{\text{Working},A} + \Prob{\text{Working}, B} + \Prob{\text{Working},C},\\
&= .2 + .278 + .208.\\
&\approx .686.
\end{align*}
In fact, that is precisely what we did above when we summed the first column of the joint probability table. We also know that any given joint probability such as $\Prob{\text{Working},C}$ is calculated by 
\begin{align*}
\Prob{\text{Working},C} &= \Prob{\text{Working}\given C} \Prob{C},\\
&=  \frac{5}{8} \times \frac{1}{3},\\
&\approx .208.
\end{align*}
Notice that this is exactly what we are doing in the tree structure calculations. Finally, when asked for $\Prob{C\given\text{Working}}$ we use the rule of conditional probability being the joint probability divided by the marginal probability, i.e. 
\begin{align*}
\Prob{C\given\text{Working}} &= \frac{\Prob{\text{Working},C}}{\Prob{\text{Working}}},\\
&= \frac{.208}{.686},\\
&\approx .304.
\end{align*}

\end{document}

